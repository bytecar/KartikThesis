\documentclass[ms,electronic,double]{nuthesis}

%% Needed to typset the math in this sample
\usepackage{amsmath}
\usepackage{amsfonts}
%% Let's use a different font
\usepackage[sc,osf]{mathpazo}

%% Makes things look better
\usepackage{microtype}

%% Makes things look better
\usepackage{booktabs}

%% Gives us extra list environments
\usepackage{paralist}

%% Be able to include graphicsx
\usepackage{graphicx}

%% If you use hyperref, you need to load memhfixc *after* it.
%% See the memoir docs for details.
\usepackage[%
pdfauthor={Ned W. Hummel},
pdftitle={Test Thesis},
pdfsubject={Thesis},
pdfkeywords={LaTeX, Thesis, University of Nebraska, Test},
linkcolor=dark-blue,
pagecolor=dark-green,
citecolor=dark-blue,
urlcolor=dark-red,
colorlinks=true,
backref,
plainpages=false,% This helps to fix the issue with hyperref with page numbering
pdfpagelabels% This helps to fix the issue with hyperref with page numbering
]{hyperref}

%% Needed by memoir to fix things with hyperref
\usepackage{memhfixc}

%% I like darker colors
\usepackage{color}
\definecolor{dark-red}{rgb}{0.6,0,0}
\definecolor{dark-green}{rgb}{0,0.6,0}
\definecolor{dark-blue}{rgb}{0,0,0.6}

\begin{document}
\frontmatter
\title{Intelligent Job Router}
\author{Kartik Vedalaveni}
\adviser{Dr. David Swanson}
\adviserAbstract{Dr. David Swanson}
\major{Computer Science}
\degreemonth{May}
\degreeyear{2013}
\college{Graduate College}
\university{University Of Nebraska}
\city{Lincoln}
\state{Nebraska}
\doctype{Thesis}
\degree{Master Of Science}
\degreeabbreviation{M.S}
\maketitle

\begin{abstract}
Schedulers come with plethora of features and 
options for customization that fulfills myriad goals of  
clusters and data centers . Often there is a need to extend these schedulers to solve 
situations arising from new use cases. One such case is when any resource like I/O, RAM or network is
throttled and the degradation occurs as a result of it . With increase in number of entities concurrently 
using the resource, there is a 
need to monitor and schedule concurrent and unscrupulous access to any given resource.
These issues that we encounter in real life at Holland Computing Center (CITE) are the 
basis and motivation for tackling this problem with a goal to run clusters and data centers
at high efficiency. The end result is the development of a Co-Scheduler for the
cluster that minimizes if not solves the problem of excessive performance degradation.

\end{abstract}

\begin{dedication}
Dedicated to 
\end{dedication}

\begin{acknowledgments}
Thanks
\end{acknowledgments}

\tableofcontents
%\listoffigures
%\listoftables


%%   mainmatter is needed after the ToC, (LoF, and LoT) to set the
%%   page numbering correctly for the main body
\mainmatter

\chapter{Introduction}
Modern schedulers in clusters provide innumerable features, the problem of cluster 
performance degradation that occurs due to improper load balancing is a problem  that hasn't been addressed. 
The problem of performance degradation when many jobs are scheduled on single system either 
based on processor equivalence or based on number of processor slots. Some of 
these schedulers like maui (CITE) are smart enough to take into account contention of other 
resources like RAM but ultimately convert the 2D vector values of CPU and RAM 
into a single scalar value which equals to hard-coding the value or presenting these resources
in some kind of ratio which makes us question effectiveness of such scheduling mechanism.


\section{Co-Scheduler}

DHTC environment


%% Thesis goes here
\chapter{Background}

\section{HTCondor} HTCondor is a distributed system developed by HTCondor team at the 
University of Wisconsin-Madison. It provides High-Throughput Computing environment to sites 
that foster research computing and enables sites to share computing resources when 
computers are idle at a given site. HTCondor system includes a batch queuing 
system for a pool of computers mainly used for compute-intensive jobs, HTCondor runs on both
 UNIX and windows based workstations that are all connected by a network.  
 HTCondor serves the research community by providing them a queuing mechanism, 
 scheduling policy, priority scheme and resource classification. Although there are other 
 batch schedulers out there for dedicated machines. The power of condor comes from 
 the fact that  the amount of compute power represented by sum total of all the 
 non-dedicated desktop workstations sitting on people's desks is sometimes far 
 greater than the compute power of dedicated central resource. There are many 
 unique tools and capabilities in HTCondor which make utilizing resources from 
 non-dedicated systems effective. These capabilities include process checkpoint 
 and migration, remote system calls and ClassAds.


\section{High-Throughput Computing} High Throughput Computing, HTC is defined as 
a computing environment that that delivers large amounts of computational
power over a long period of time.  The important factor being over a long period of time which 
differentiates HTC from HPC which focuses on getting large amount of work done in small amount of time.
The workloads that run on condor system doesn't have an objective of  how fast the job can be completed 
but how many times can the job be run in the next few months.In another definition of HTC, European Grid  
Infrastructure defines HTC as a computing paradigm that focuses on the efficient 
execution of large number of loosely coupled tasks.

\section{Open Science Grid} Open Science Grid(OSG), provides service and support 
for resource providers and scientific institutions using a distributed fabric of 
high throughout computational services. OSG was created to facilitate data analysis from the 
Large Hadron Collider . OSG doesn't own resources but provides software and services to 
users and enables opportunistic usage and sharing of resources among resource providers.
The main goal of OSG is to advance science through open distributed computing. 
The OSG provides multi-disciplinary partnership to federate local, regional, community and 
national cyber-infrastructures to meet the needs of research and academic communities at all scales.

OSG provides resources and directions to Virtual Organizations(VO's) for the purposes of LHC experiments
and HTC in general. \\

Building a OSG site requires listing background and careful planning. The major 
components of a OSG site includes a Storage Element and Compute Element. \\

Storage elements (SE) manage physical systems, disk caches and hierarchical mass storage 
systems, its an interface for grid jobs to underlying storage Storage Resource Management protocol and Globus 
Grid FTP protocol and others, A storage element requires an underlying storage system like hadoop, xrootd
and a GridFTP server and an SRM interface.\\

A Compute Element(CE) allows grid users to run jobs on your site. It provides a 
bunch of services when run on the gatekeeper. The basic components include 
the GRAM and GridFTP on the same CE host to successfully enable file transfer 
mechanisms of Condor-G.\\

\section{SPC}

\section{Job Router}
Condor Job Router as defined in condor manual\cite{manual56} transforms jobs from vanilla 
universe to grid universe according to a configurable policy.   Condor Job 
Router helps to balance jobs across clusters by transferring excess jobs from 
one cluster to another.  The rate of job submissions equals the rate at which 
site starts running the job. The other mechanisms including glidein and condor 
flocking does not provide this kind of balancing mechanism.

High throughput work flows benefits a lot from Job routing as its easy to reach the goal of distributing the 
workload and keeping as many computers as busy as possible. The Job Router does not know which site
will run the jobs faster but it can decide whether to send more jobs to a site based on whether the already 
submitted Jobs are sitting idle or not or whether the site has experienced 
recent job failures.

%% backmatter is needed at the end of the main body of your thesis to
%% set up page numbering correctly for the remainder of the thesis
\backmatter

%% Start the correct formatting for the appendices
\appendix

%% Appendices go here (if you have them)

%% Bibliography goes here (You better have one)
%% BibTeX is your friend
%% Index go here (if you have one)
%% Bibliography goes here (You better have one)
%% BibTeX is your friend
\bibliographystyle{plain}
\bibliography{KartikThesis}
%% Pull in all the entries in the bibtex file. Is is a useful trick to
%% check all your references.
\nocite{*}

%% Index go here (if you have one)

\end{document}