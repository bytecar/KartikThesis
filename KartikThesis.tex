\documentclass[ms,electronic,double]{nuthesis}

%% Needed to typset the math in this sample
\usepackage{amsmath}
\usepackage{amsfonts}
%% Let's use a different font
\usepackage[sc,osf]{mathpazo}

%% Makes things look better
\usepackage{microtype}

%% Makes things look better
\usepackage{booktabs}

%% Gives us extra list environments
\usepackage{paralist}

%% Be able to include graphicsx
\usepackage{graphicx}

%% If you use hyperref, you need to load memhfixc *after* it.
%% See the memoir docs for details.
\usepackage[%
pdfauthor={Ned W. Hummel},
pdftitle={Test Thesis},
pdfsubject={Thesis},
pdfkeywords={LaTeX, Thesis, University of Nebrska, Test},
linkcolor=dark-blue,
pagecolor=dark-green,
citecolor=dark-blue,
urlcolor=dark-red,
colorlinks=true,
backref,
plainpages=false,% This helps to fix the issue with hyperref with page numbering
pdfpagelabels% This helps to fix the issue with hyperref with page numbering
]{hyperref}

%% Needed by memoir to fix things with hyperref
\usepackage{memhfixc}

%% I like darker colors
\usepackage{color}
\definecolor{dark-red}{rgb}{0.6,0,0}
\definecolor{dark-green}{rgb}{0,0.6,0}
\definecolor{dark-blue}{rgb}{0,0,0.6}

\begin{document}
  
\frontmatter
\title{Intelligent Job Router}
\author{Kartik Vedalaveni}
\adviser{Dr. David Swanson}
\adviserAbstract{Dr. David Swanson}
\major{Computer Science}
\degreemonth{May}
\degreeyear{2013}
\college{Graduate College}
\university{University Of Nebraska}
\city{Lincoln}
\state{Nebraska}
\doctype{Thesis}
\degree{Master Of Science}
\degreeabbreviation{M.S}

\maketitle

\begin{abstract}

Most schedulers available come with plethora of features and 
options for customizations that fulfills myriad goals of a custom built 
clusters and data centers but there are some important functions required to
run clusters at its peak efficiency. One such case is when filesystem crashes over large I/O 
workload and we see performance of the cluster degrading when there is absence of balance in load
pertaining to either I/O, network, CPU or RAM usage.
These issues that we encounter in real life at Holland Computing Center are the 
basis and motivation for tackling these set of problems with a goal to run clusters and data centers
at high efficiency. The end result is the development of a Co-Scheduler for the
cluster that minimizes if not solves the problem of performance degradation that is caused when some of the resource 
is throttled to a greater extent for some kind of workloads.

\end{abstract}

\begin{dedication}
Dedicated to 
\end{dedication}

\begin{acknowledgments}
Thanks
\end{acknowledgments}

\tableofcontents
%\listoffigures
%\listoftables


%%   mainmatter is needed after the ToC, (LoF, and LoT) to set the
%%   page numbering correctly for the main body
\mainmatter

\chapter{Introduction}
Although modern schedulers in clusters provide innumerable features, the problem of cluster 
performance degradation that occurs due to improper load balancing hasn't been addressed yet. The problem of 
performance degradation when many jobs are scheduled on single system either 
based on processor equivalence or based on number of processor slots. Some of 
these schedulers are smart enough to take into account contention of other 
resources like RAM but ultimately convert the 2D vector values of CPU and RAM 
into a single scalar value which equals to hard-coding the value or presenting these resources
in some kind of ratio makes us question effectiveness of such scheduling mechanism.

\section{Co-Scheduler}


%% Thesis goes here
\chapter{Background}
HTCondor is a distributed system developed by HTCondor team at the University of 
Wisconsin-Madison.It provides High-Throughput Computing environment to sites 
that foster research computing and enables sites to share computing resources when 
workstations are idle at a given site. HTCondor system includes a batch queuing 
system for a pool of workstations, HTCondor runs on both UNIX and windows based 
workstations that are all connected by a network. 

\section{High-Throughput Computing} The workloads that run on 
condor system doesn't have an objective of  how fast the job can be completed but how many 
times can the job be run in the next few months.To be precise, European Grid 
Infrastructure defines HTC as a computing paradigm that focuses on the efficient 
execution of large number of loosely coupled tasks.

\section{Open Science Grid} Open Science Grid(OSG), provides service and support 
for resource providers and scientific institutions using a distributed fabric of 
high throughout computational services. Initially OSG was created to facilitate data analysis from the 
Large Hadron Collider and about 70\% of the resources are used on analysis of data from particle colliders.
OSG doesn't own resources but provides software and services to users and enables opportunistic usage and  sharing of resources among
resource providers.The main goal of OSG is to advance science through open 
distributed computing. The OSG provides multi-disciplinary partnership to federate 
local, regional, community and national cyber-infrastructures to meet the needs 
of research and academic communities at all scales. 



%% backmatter is needed at the end of the main body of your thesis to
%% set up page numbering correctly for the remainder of the thesis
\backmatter

%% Start the correct formatting for the appendices
\appendix

%% Appendices go here (if you have them)
%% Bibliography goes here (You better have one)
%% BibTeX is your friend
%% Index go here (if you have one)

%% Bibliography goes here (You better have one)
%% BibTeX is your friend
\bibliographystyle{plain}
\bibliography{nuthesis}
%% Pull in all the entries in the bibtex file. Is is a useful trick to
%% check all your references.
\nocite{*}

%% Index go here (if you have one)



\end{document}