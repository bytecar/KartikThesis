%% print or electronic
\documentclass[ms,electronic,double]{nuthesis}

%% Needed to typset the math in this sample
\usepackage{amsmath}
\usepackage{amsfonts}
%% Let's use a different font
\usepackage[sc,osf]{mathpazo}

%% Makes things look better
\usepackage{microtype}

%% Makes things look better
\usepackage{booktabs}

%% Gives us extra list environments
\usepackage{paralist}

%% Be able to include graphicsx
\usepackage{graphicx}
%% Needed to display code 
\usepackage{listings}
\usepackage{algorithmic}
\usepackage{algorithm}
\usepackage{subfig}
\usepackage{url}
\usepackage{placeins}

%% I like darker colors
\usepackage{color}
\definecolor{dark-red}{rgb}{0.6,0,0}
\definecolor{dark-green}{rgb}{0,0.6,0}
\definecolor{dark-blue}{rgb}{0,0,0.6}
\definecolor{dark-purple}{rgb}{0.5,0,0.35} 
\definecolor{light-grey}{rgb}{0.9,0.9,0.9}

%% If you use hyperref, you need to load memhfixc *after* it.
%% See the memoir docs for details.
\usepackage[%
pdfauthor={Kartik Vedalaveni},
pdftitle={Adaptive Co-Scheduler for highly Dynamic Resources},
pdfsubject={Thesis},
pdfkeywords={LaTeX, Thesis, University of Nebraska, Intelligent Co-Scheduler},
linkcolor=dark-blue,
pagecolor=dark-green,
citecolor=dark-blue,
urlcolor=dark-red,
colorlinks=true,
backref,
plainpages=false,% This helps to fix the issue with hyperref with page numbering
pdfpagelabels% This helps to fix the issue with hyperref with page numbering
]{hyperref}

%Listings color code
\lstset{
language=c,
basicstyle=\ttfamily,
keywordstyle=\color{dark-purple}\bfseries,
stringstyle=\color{dark-red},
commentstyle=\color{dark-green},
morecomment=[s][\color{dark-blue}]{/**}{*/},
numbers=left,%left, right, none
numberstyle=\tiny\color{black},
stepnumber=1,
numbersep=10pt,
tabsize=4,
showspaces=false,
showstringspaces=false,
frame=single,
backgroundcolor=\color{light-grey}
}

%% Needed by memoir to fix things with hyperref
\usepackage{memhfixc}


\begin{document}
\frontmatter
\title{Adaptive Co-Scheduler for highly Dynamic Resources}
\author{Kartik Vedalaveni}
\adviser{Professor David Swanson}
\adviserAbstract{David Swanson}
\major{Computer Science}
\degreemonth{August}
\degreeyear{2013}
\college{Graduate College}
\university{University Of Nebraska}
\city{Lincoln}
\state{Nebraska}
\doctype{Thesis}
\degree{Master Of Science}
\degreeabbreviation{M.S}
\maketitle

\begin{abstract}
 There are many kinds of scientific applications that run on high throughput computational (HTC) 
grids. HTC may utilize clusters opportunistically, only running on a given cluster 
when it is otherwise idle. These widely dispersed HTC clusters are heterogeneous in terms of 
capability and availability, but can provide significant computing power in aggregate. The scientific 
algorithms run on them also vary greatly. Some scientific algorithms might use high rates of disk I/O 
and some might need large amounts of RAM. Most schedulers only consider cpu availability, but 
unchecked demand on these associated resources that aren't managed by resource managers 
may give rise to several issues on the cluster.

On the grid there could be different schedulers on different sites and we cannot rely upon features 
of one kind of scheduler to characterize the nature and features of every job. Most state of the art 
schedulers do not take into account resources like RAM, Disk I/O or Network I/O. This is as true for the
 local schedulers as much it is for the grid. Often there is a need to extend these schedulers to solve 
 situations arising from new and/or complex use cases either by writing a plugin for existing schedulers or by Co­Scheduling. 
 A key issue is when resources like RAM, Disk I/O or Network I/O are used in an unchecked manner and 
 performance degrades as a result of it. Further scheduling jobs that claim the degraded resources could 
 overwhelm the resource to an extent that the resource will finally stop responding or theWe solve system will crash. 
 
 We schedule based on minimum turnaround time of the sites which will help increase 
the throughput of the overall workload for the Co-Scheduler, which also is a good load-balancer in itself. 
The Co-Scheduler is driven by the fact that turnaround time increases when 
concurrent jobs accessing these resources reach a threshold value which in-turn causes 
degradation and this is the basis for this work.

With an increase in the number of entities concurrently using the resource, there is a need to monitor and schedule concurrent and unmanaged access to any given resource to prevent
 degradation. These issues that we encounter in real life at the Holland Computing Center 
are the basis and motivation for tackling this problem and for developing an adaptive approach for scheduling. 
This co-scheduler must be aware of multi­-resource degradation, balance load across multiple sites and 
run clusters at high efficiency and share resources fluidly. An initial implementation tested at HCC will 
be evaluated and presented. 

  
\end{abstract}

%\begin{dedication}
%Dedicated to 
%\end{dedication}

\begin{acknowledgments}

First and foremost, I would like to thank my advisor Dr. David Swanson. Throughout my research
and writing he's been a constant source of support for me. His guidance and teachings have helped not 
only to shape my thesis but also in all walks of my life. I thank David for identifying this interesting
research problem and bringing it to me. His hard work has been a constant source of inspiration.
 I would like to thank Derek Weitzel for working with me to solve the 
 issues with grid submissions and for helping me with his technical expertise.
 
 I'd like to take this opportunity to thank the entire team at Holland 
Computing Center for providing me with the required resources, support and continuous advise
 even after I broke their systems. It has been a wonderful experience 
 working with these folks.
 
\end{acknowledgments}

\tableofcontents
\newpage
\listoffigures
%\listoftables


%%   mainmatter is needed after the ToC, (LoF, and LoT) to set the
%%   page numbering correctly for the main body
\mainmatter

\chapter{Introduction}
Ian Foster states in his paper The Anatomy Of the Grid \cite{Foster:2001:AGE:1080644.1080667} that 
\emph{Grid computing is about controlled sharing of resources with resource owners enforcing policies 
on the owned resources.} The resources come in the form of hardware 
and software that allows us to submit jobs, run the jobs and monitor the jobs on the grid. 
Universities usually have multiple clusters across their campus and these are 
usually owned by different departments but stand united under the banner of the university.
Campus grids are mini grids where jobs are spanned across multiple clusters
 based on the need of the user and available resources. There is a provision for jobs to overflow to the national
grid infrastructure\cite{derekThesis}.

Modern schedulers used in clusters provide numerable features for policy making, resource management 
and scheduling. The problem of cluster 
performance degradation that occurs when one of the resources is overwhelmed is a problem that hasn't been 
addressed. This problem occurs when many jobs are scheduled on a single system. Some of 
the schedulers like maui \cite{pbstorque} are sophisticated enough to take into account contention of other 
resources like RAM but ultimately convert the 2D vector values of CPU and RAM 
into a single scalar value. In practice, this hard-codes the value or presents these resources
in a fixed ratio which makes us question the effectiveness of such scheduling mechanisms. 

At Holland Computing Center of the University of Nebraska-Lincoln, we're 
tackling this issue of cluster degradation caused by 
over exploitation of one or more resources. 
The proposed solution adaptively schedules 
such highly dynamic resources on the grid and across multiple sites by adaptively scaling with respect to
the performance of a given cluster. We schedule based on minimum turnaround time of the sites which will help increase 
the throughput of the overall workload for the Co-Scheduler, which also is a good load-balancer in itself.

Existing schedulers depend on the availability of resources and frequent polling 
of it via resource manager to determine the slots for scheduling. It should be noted that state of the art 
schedulers like maui/torque, slurm, HTCondor take into account only CPU as a 
resource and resources like RAM, Disk I/O or Network I/O are either ignored or 
their resource equivalent is converted into a scalar value which isn't an effective way of tackling the 
multiple resource scheduling problem. This might result in scheduling excessive jobs 
on a single machine. In some schedulers, for example, it will be pre-assigned on every node that 1 CPU-CORE will have 2 GB of
memory. This kind of pre-assignment seems like it is addressing the resource degradation of RAM in addition to CPU but suffers 
from the above mentioned problems. We take a turnaround time approach to measure degradation. We define a resource 
to be degraded if and when we submit jobs that result in increasing the 
turnaround time by 25\% . It must be 
noted that there isn't explicit measurement of status of individual resources like RAM, Disk I/O, 
Network I/O, since there are no guarantees the resource managers keep track of these resource allocations. 
The Co-Scheduler is driven by the fact that turnaround time increases when 
concurrent jobs accessing these resources reach a threshold value which in-turn causes 
degradation and this is the basis for this work.

One aspect to Co-Scheduler is degradation detection and management by adapting 
to the throttling of a resource. Another aspect is to efficiently distribute jobs 
across multiple sites which increases the throughput of the workload. The Co-Scheduler adapts to degradation by backing off submission rates and waiting for a 
period of time (till the detection of a target capacity) for the next incremental submission. It efficiently 
distributes the jobs and balances load across multiple sites, thus submitting more 
jobs to a site with less turnaround time and also ensuring to submit lesser 
jobs to the sites with larger turnaround time. This efficiently balances load 
across multiple sites on a grid and improves throughput because the loads may vary dynamically. 
The capacity of jobs the cluster can efficiently run without degradation needs to be detected and updated again and 
again after a given period of time.

Finally, since the grid environment is a heterogeneous environment, the scheduler implementation must use APIs that are available on all the systems across the grid. To implement such a Co-Scheduler
 we limit ourselves to HTCondor based clusters and utilize libcondorapi that needs to be present on the submission site and we can still submit jobs to sites without HTCondor scheduler via $Condor\_G$ interface. 
 The result is a threaded Co-Scheduler 
that submits to multiple sites concurrently, is aware of multiple-resource degradation, and
balances load efficiently across multiple sites of the grid. 

Please note that all the references to the Co-Scheduler in this thesis refers to the adaptive Co-Scheduler designed 
by us.  

%% Thesis goes here
\chapter{Background}

\section{High-Throughput Computing} High Throughput Computing (HTC) is defined as 
a computing environment that delivers large amounts of computational
power over a long period of time. The important factor being over a long period of time which 
differentiates HTC from HPC which focuses on getting a large amount of work done in a small amount of time.
The workloads that run on HTCondor systems don't have an objective of  how fast the job can be completed 
but how many times can the job be run in the next few months. In another definition of HTC, European Grid  
Infrastructure defines HTC as a computing paradigm that focuses on the efficient 
execution of a large number of loosely coupled tasks \cite{manual56}. One such 
distributed computing software providing High-Throughput Computing services is 
HTCondor developed by the team at University of Wisconsin-Madison\cite{manual56}.


\section{HTCondor} HTCondor is a distributed system developed by HTCondor team at the 
University of Wisconsin-Madison. It provides an HTC environment to sites 
that foster research computing and enables sites to share computing resources on otherwise  
idle computers at a given site. HTCondor includes a batch queuing 
system, scheduling policy, priority scheme, and resource classifications for a pool of 
computers, which is mainly used for compute-intensive jobs. HTCondor runs on both
 UNIX and Windows based workstations that are all connected by a network, although there are other batch schedulers out there for dedicated machines. 
The power of HTCondor comes from  the fact that  the amount of compute power 
represented by the sum total of all the 
 non-dedicated desktop workstations sitting on people's desks is sometimes far 
 greater than the compute power of a dedicated central resource. There are many 
 unique tools and capabilities in HTCondor which make utilizing resources from 
 non-dedicated systems effective. These capabilities include process checkpoint 
 and migration, remote system calls and ClassAds. HTCondor also includes a 
 powerful resource manager with an efficient match-making mechanism that is 
 implemented via ClassAds, which makes HTCondor understandable when compared with other 
 compute schedulers\cite{manual56}. With the numerous features of HTCondor, it is 
 used for grid computing on the scale of a large number of massive computers 
 that are loosely coupled on the open science grid. We architected a 
 bioinformatics program iSG, indel sequence generator, which simulates the evolutionary events of highly diverged DNA and protein 
 sequences. iSG is a high memory footprint program. We architected it so as to 
 make it run on the grid and use less amount of peak memory \cite{condorisg_paper}. 
 We retained the order of execution of individual jobs 
 by making extensive use of HTCondor's DAGMAN. DAGMAN allows one to manage a 
 progressing of ordered execution of jobs, the jobs are in the form of a 
 directed acyclic graph.
 
 
\section{Grid Computing}

As the cost of computers and the network connecting them is decreasing, there is movement towards a 
paradigm shift in computing with the clustering of geographically distributed 
resources. The goal of this is to provide a 
service oriented infrastructure. Organizations like the Grid community and Global Grid Forum are constantly investing effort in making Grid a platform with standard 
protocols \cite{Foster:2001:AGE:1080644.1080667}, to provide seamless and secure discovery and access to infrastructure and interactions among
 the resources and services.
 
 With the advent of parallel programming and distributed systems it became 
 obvious that loosely coupled computers can be used for computing purposes and this 
 network of workstations  gave rise to the notion of distributed computing. 
 The Globus project began in 1996 and Argonne National Laboratory was responsible for a process 
 and middleware communication system called Nexus which provides remote service 
 requests across heterogeneous machines. The goal of Globus was to build a 
 global Nexus that would provide support for resource discovery, data access, 
 authentication and authorization.

At this point ``Grid" replaced the use of ``meta-computer" and \emph{ researchers from 
a collaboration of universities termed Grid as integrated resources 
with integration of many computational visualization and information resources 
into a coherent infrastructure.}

The following are goals and visions of Grid Computing:


\begin{description}
  \item[Seamless Aggregation of Resources and Services]
  Aggregation involves three aspects, 
  \begin{enumerate}
    \item{Aggregation of geographically distributed resources}
    \item{Aggregation of capacity}
    \item{Aggregation of capability}
      \end{enumerate}
    The key factors to enable such aggregation includes protocols and mechanisms 
    to secure discovery, access to and aggregation of resources for the 
    realization of virtual organization and applications that can exploit such 
    an environment.

  \item[Ubiquitous Service-Oriented Architecture] This is the ability of the grid 
  environment to do secure and scalable resource and data discovery along with scheduling and 
  management based on a wide variety of application domains and styles of 
  computing.
  
  \item[Autonomic Behaviors]
  The dynamism, heterogeneity and complexity of the grid has made many 
  researchers rethink their systems. This new trend aims at system configuration 
  and maintenance of the grid with the least human effort and has led to numerous 
  projects like autonomic grids, cognitive grids and semantic grids.
  
\end{description}

A key computational grid in the USA is the Open Science Grid, described further 
in section 2.5.
At the heart of the grid computing tools provided by the open science grid lies the globus toolkit, that provides
features and interfaces that enable grid computing. 
\section{Globus}
The Globus project\cite{globus} was intended to accelerate the then meta-computing that was built on 
distributed and parallel software technologies. The term meta-computing was used 
before the term grid was actually coined which refers to 
a networked virtual supercomputer, constructed dynamically from geographically 
distributed resources linked by high speed networks \cite{globus}. Scheduling 
and managing a large group of heterogeneous resources is a challenging and 
daunting task on the grid. The Globus toolkit provides a framework for managing 
and scheduling of grid resources across heterogeneous environments.

The Globus toolkit consists of a set of modules,  each module provides an 
interface for the provision of an implementation of low level functionalities of the toolkit:

\begin{itemize}
  \item{Resource location and allocation} 
  \item{Communications}
  \item{Unified resource information service}
  \item{Authentication interface}
  \item{Process creation}
  \item{Data Access}
\end{itemize}

There are multiple Virtual Organizations across Open Science Grid. One such VO 
is Nebraska's HCC VO at Holland Computing Center. A VO provides researchers with 
computing resources and enables sharing of resources across other VO's. It additionally signs the user's certificate so that a user is identified with a given VO.

We delegate our identification on the grid. \emph{voms-proxy-init}
is used to generate a voms proxy at the submit host, we can specify voms as our 
virtual organization, hcc:/hcc in this case, and hours as the number of hours the proxy would be 
active/valid. It must be ensured that the proxy period is approximately greater than the 
length of period of runtime of jobs for successful running of all the jobs. A 
proxy provides a secure way to access grid resources, by creating a proxy and 
delegating our identity to it for a short period of time. It enables us to sandbox 
the theft of the proxy credentials and limit the damage to the short lived proxy 
and keep our original set of credentials safe.

HTCondor-G extends HTCondor to the grid environment. Jobs are sent across the grid universe using 
Globus software. The Globus toolkit provides support for building grid systems but 
submitting, managing and executing jobs have the same capabilities in both the HTCondor 
and the HTCondor-G world. Globus provides fault tolerant features to HTCondor-G 
jobs. GRAM is the Grid Resource Allocation and Management protocol, which supports remote 
submission of computational request.

gt2 is an initial GRAM protocol which is used in Globus Toolkit version 1 and 
2. gt2 is also referred to as the pre-web services GRAM or GRAM2. Similarly, gt5 is the latest GRAM protocol, which is an extension of GRAM2 and is intended 
to be more scalable and robust, referred to as GRAM5.

\section{Open Science Grid} The Open Science Grid(OSG) \cite{osg}, provides service and support 
for resource providers and scientific institutions using a distributed fabric of 
high throughout computational services. OSG was created to facilitate data analysis from the 
Large Hadron Collider\cite{osg}. 
OSG provides resources and directions to Virtual Organizations(VO's) for the purposes of LHC experiments
and HTC in general.OSG doesn't own resources but provides software and services to 
users and enables opportunistic usage and sharing of resources among resource providers.
The main goal of OSG is to advance science through open distributed computing. 
The OSG provides a multi-disciplinary partnership to federate local, regional, community and 
national cyber-infrastructures to meet the needs of research and academic communities at all scales.

Building an OSG site requires analysis of the requirements and careful planning. The major 
components of an OSG site include a Storage Element and a Compute Element. \\
Storage elements (SE) manage physical systems, disk caches and hierarchical mass storage 
systems. A SE is an interface for grid jobs to the Storage Resource Management protocol (SRM) and the Globus 
GridFTP protocol and others. A storage element requires an underlying storage system like hadoop, xrootd
and a GridFTP server and an SRM interface.\\
A Compute Element(CE) allows grid users to run jobs on your site. It provides a 
large number of services when run on the gatekeeper. The basic components include 
the GRAM and GridFTP on the same CE host to successfully enable file transfer 
mechanisms of HTCondor-G.\\


\chapter{Related Work}

\section{Comparison of existing mechanisms}
There are situations where we can have multiple HTCondor pools exist and some of 
the pools have many idle slots that are available for utilization. To efficiently utilize the resources across pools condor provides 
mechanisms like HTCondor flocking and HTCondor job router. These mechanisms 
provide similar functionality to the Co-Scheduler designed here. In the following sections, we compare 
and contrast these mechanisms with the design and functionality of Co-Scheduler in grid environments.

\subsection{HTCondor Flocking}
Flocking refers to a mechanism where a job that cannot run on its own HTCondor pool 
due to a lack of resources runs in another HTCondor pool where resources are 
available. HTCondor flocking enables load sharing between pools of computers. As 
pointed out by Campus Grids Thesis\cite{derekThesis} flocking helps balance large workflows across different pools because of the scavenging and greedy nature 
of the HTCondor scheduler.

To accomplish flocking HTCondor uses multiple components. 
\texttt{condor\_schedd} advertises that it has idle jobs to the remote \texttt{condor\_collector}. During the next phase of negotiation, if it's found that there are computers 
available then they are allotted to the jobs in the matchmaking phase and the 
jobs are then run on the remote pools. It so appears, and the local job queue's maintained as if 
the jobs are running locally. 

Although HTCondor flocking has workflow balancing features across multiple HTCondor 
pools, it isn't aware of the slower and faster sites/pools. The adaptive Co-Scheduler 
keeps tabs on turnaround time and is aware of which site is faster or slower. If we 
look at the idea that scavenging idle resources increases throughput, it does. However, if we look at the case where jobs are greater than the available slots 
across all the pools, HTCondor flocking stops working here as deeper understanding of sites
would be required here to push more jobs at faster sites and importantly push less jobs
at slower sites. Flocking's role is to scavenge idle computing slots across multiple pools. 

Another aspect where HTCondor flocking isn't designed to perform well is when jobs are 
contending for the same resource (CPU, RAM, Disk I/O \& Network I/O). Even though there will be 
idle slots on remote pools, HTCondor flocking might not be able to use those 
slots as they'd be degraded because of the contention. In this case again 
Co-Scheduler comes in handy. Based on the turnaround time, Co-Scheduler relaxes the load on the resource by submitting fewer  jobs 
this helps reduce degradation and doesn't put excessive job load on the 
degraded resource. It diverts the jobs elsewhere to perhaps another site, which in-turn increases the 
throughput.

To conclude, we can say that HTCondor flocking provides features of balancing 
large workflows and doesn't include features that would detect degradation in 
the cluster. It also doesn't keep track of information of site performance, which might be exploited to increase the overall throughput of the system. 

\subsection{HTCondor Job Router}

The HTCondor manual defines the functions of job router to be the following:
\begin{quotation}

The HTCondor Job Router is an add-on to the \texttt{condor\_schedd} that transforms jobs from one type into 
another according to a configurable policy\cite{manual56}. 
This process of transforming the jobs is called job routing.
\end{quotation}

HTCondor Job Router can transform vanilla universe jobs to grid universe jobs and 
as it submits to multiple sites, the rate at which it starts submitting equals 
the rate at which the sites execute them. This provides a platform to balance large 
workflows across multiple grid sites and replenish the jobs at
faster sites once they get done. Job router sends more jobs to a site if 
the jobs submitted are not idle and stops submitting jobs if the submitted jobs 
sit idle on the remote cluster. Job router is not aware about which site is 
faster of slower.


\begin{figure}[htbp!]
\begin{center}
\includegraphics[scale=0.75]{images/jobRouter}
\caption{JobRouter: Transformation of Jobs}
\label{fig:JobRouter}
\end{center}
\end{figure}

A job is transformed to the grid universe by making a copy of the original job 
ClassAd and modifying some attributes of the job. This copy is called the routed 
copy and this routed copy shows up in the job queue with a new job id\cite{manual56}.

HTCondor job router utilizes a routing table which contains the listing of sites 
the job must be submitted to and the name of the potential grid resources. Routing is processed via a  HTCondor config file which is defined by the new ClassAds.

\begin{figure}
\begin{lstlisting}
  # Now we define each of the routes to send jobs on
JOB_ROUTER_ENTRIES = \
   [ GridResource = "gt5 ff-grid.unl.edu/jobmanager-pbs"; \
     name = "Firefly"; \
   ] \
   [ GridResource = "gt5 tusker-gw1.unl.edu/jobmanager-pbs"; \
     name = "Tusker"; \
   ] \
   [ GridResource = "gt5 pf-grid.unl.edu/jobmanager-condor"; \
     name = "Prairiefire"; \
   ]\

\end{lstlisting}
\caption{Configuration to implement Job Routing}
\end{figure}

HTCondor Job Router appears to be a step up from HTCondor Flocking in terms of 
scavenging resources and sending the extra jobs to another HTCondor pool. HTCondor 
Job Router also maintains the submission rate of the jobs on submit hosts equal to that of remote clusters. But the job 
router does not keep track of how fast each HTCondor cluster is. Thus 
HTCondor job router does not optimize the displacement of jobs from a slower 
cluster. Since it maintains the rate of submission of jobs, we can be sure that if a job is completed 
at a faster site, it's immediately replaced by the next one. The same is true for a job at a slower site.
This might result in increasing the degradation if there exists some. For example, if the job router kept track of slower sites, it
could send less jobs to slower sites and thereby increase the overall throughput of the given workflow.

A preliminary examination shows that HTCondor job router does seem to have degradation detection features, 
since it does not submit to a pool that already has idle jobs. However, close examination reveals that even though it
submits to a pool with non-idle jobs, it isn't aware of the fact that these jobs in the queue would have 
undergone degradation due to contention on some resource and thus isn't a viable mechanism for degradation detection.

\subsection{OSG Match Maker}

OSG Match maker is a tool that was created to be distributed among small to 
medium sized VOs to be used as a powerful submit interface to OSG. It is 
designed to retrieve VO specific site information and also to verify and 
maintain the sites by regularly submitting verification jobs to make sure a site 
can continue receiving jobs.
OSG Match maker is another tool that provides information about the grid sites to the 
HTCondor scheduler. HTCondor scheduler does all the matchmaking with the ClassAds attributes. 
OSG Match Maker is no longer supported on OSG.
  

\chapter{Adaptive Co-Scheduler For Highly Dynamic Resources}

\section{Introduction}
Due to the heterogeneous nature of the grid it is challenging to schedule the dynamic resources on the grid. 
We propose a solution that primarily solves the problem of degradation due to contention among 
jobs for any given resource like RAM, I/O and the network. The contention among jobs for the same given
resource gives rise to degradation. For example, contention on large scale job submissions can crash the I/O resource.
 Efficient and high throughput
distribution of jobs across multiple clusters is another challenging problem that we have addressed in 
this particular Co-Scheduler Solution. 

\begin{figure}[htbp!]
\begin{center}
\includegraphics[scale=0.75]{images/degradation_detection}
\caption{Degradation occurrence: Co-Scheduling}
\label{fig:degradationdetect-intro}
\end{center}
\end{figure}
\FloatBarrier

In figure \ref{fig:degradationdetect-intro}, with C1 called job propagation factor and k the initial number of 
jobs submitted, we can see that on each iteration $k = C1 * k$ number of jobs are added. However, if we have a degraded system, we submit $k=k/C1$ number of 
jobs by backing off on the amount of jobs submitted.

\section{Degradation detection and Capacity based scheduling}
In the first part of the solution we target the basic problem of degradation. To give an example, suppose
we have a file server that can serve 100 MBps of data which is available for concurrent access and a user has submitted 200 jobs
each consuming 10 MBps of data I/O. If 200 CPU slots are available, a typical state of the art Scheduler
schedules these 200 jobs without actually looking at the I/O load. This results in a degraded system 
over time if more jobs are Scheduled to utilize this file server. This file server might even crash. We would need a 
degradation handling mechanism that would adaptively scale with the load and  exponentially 
back off during a high contention period. Thus we need an intervention in the form
of a Co-Scheduler that intelligently handles degradation.
We start the capacity algorithm by submitting jobs to the cluster and by measuring 
the turnaround time of each iteration. If we find that the current turnaround 
time of an iteration is 25\% greater than the previous iteration we term it as a 
degradation. We now try to find the best capacity of jobs between the current 
and the previous iteration by exponentially backing off between these two iterations. We keep finding the 
target capacity dynamically for each iteration as the contention may change and target capacity keeps 
 varying. Once the target capacity is found then it is retained for job submissions for a given amount of time
 limited to a maximum of ten iterations of jobs, thus solving the problem of performance degradation. 

\section{Multi-site load distribution based scheduling} 
 \begin{figure}[htbp!]
   
 \begin{verbatim}
 C1 -> Job propagation constant
 multiSite -> list storing turnaround times of multiple sites.
 low ->  min(multiSite)  
 high -> max(multiSite)
 avg -> average(multiSite)
 \end{verbatim}

$$
f(C1) = \left\{
        \begin{array}{ll}
            C1*4 & \quad multiSite_i = low\\ 
            C1/2 & \quad multiSite_i = high \\
            C1*2 & \quad multiSite < avg
        \end{array}
    \right.
$$

\caption{Multi-site Job distribution function}
  \label{fig:multiSite2}

\end{figure}
Since we're already tracking site performance, the other problem that we tackled was that of efficient distribution of work load to multiple sites of the 
grid, referred in section \ref{fig:multiSite2}.
Some of these sites could be error prone, some of them could be faster and some can be slower. We need to
take an approach that improves the overall throughput of the workload, thus we keep track of average AgTime which equals (Queue wait time + Run-time) for each batch of jobs submitted to the cluster. 
We take the average AgTime of the batch of jobs that is submitted 
 across multiple clusters. We group AgTimes of these batches of jobs based on the value of the 
 AgTime to be greater than, less than, or equal to the average AgTime of all batches of jobs.
 This gives us a classification of sites that run faster and also that are more available. We exploit this information in our scheduler. 
 To implement multi-site job submission and load balancing, we start submitting one job to each cluster 
 and measure the AgTime once the job returns. Now we have the information of AgTime of all the clusters. Continuing to submit jobs concurrently in a similar manner
  would result in the completion of
 the workload in a non-efficient way. Out of the list of all the AgTimes of different sites we take the 
 sites
 that have lower than average AgTime of all batches of jobs and increase the job propagation 
 factor for these jobs. This in-turn increases the number of jobs submitted to that site and helps us efficiently distribute 
 the workload to faster clusters. Submitting to a faster site would enable us to submit 
 more jobs to the particular site. As the AgTime is lower we would finish jobs quickly and make room for more
 jobs if we submit less to slower sites. Slower sites can still help us improve the throughput to a greater extent as long as we submit less jobs to them and prevent performance degradation at those sites. The error handling mechanism is gracefully handled by 
 HTCondor and by this approach we have an efficient system with increased throughput and with proper distribution of load across multiple clusters.
 
%%A table of comparison among all three kinds of mechanisms
%\chapter{Design and Implementation of Co-Scheduler}
%(an image to showcase layer of position of co-scheduler)
The co-scheduler requires the presence of a HTCondor installation and libcondorapi.
It is written in the C++ language and has extensively made use of pthreads for 
synchronization and multithreading. The co-scheduler has two components to it: the first one is a capacity detection
algorithm which is found by measuring degradation and the other is a multi-site workload distribution algorithm. The following paragraphs detail the 
engineering aspects of the design. The program begins by taking the number of jobs, the site's information and submit script 
information as its input. The number of jobs are the ones submitted across 
multiple sites. The next input file contains information on the list of sites that can be used 
for load balancing in the GridResource format of the HTCondor ClassAds API, e.g. \emph{tusker-gw1.unl.edu/jobmanager-pbs}.
It is assumed that all the sites listed in the sites input file are working and 
do not have any misconfiguration issues. The third and the final option to the 
scheduler is the submit description file. All the job ClassAd information and 
requirements can be written in this section and all of these will be applicable on the grid when a particular job is 
scheduled. There are many different kinds of universe in condor including Vanilla, java, grid. vanilla enables job submissions to the local
condor pool, java enables submissions of java programs and grid enables direct submission to the grid. 
The two main important attributes required in the submit description 
file are \emph{universe} which needs to be ``grid" all the time as we are submitting 
to grid sites via HTCondor-G, and the other most important thing is the \emph{grid-proxy} which is elaborated in section 2.4. 
\texttt{condor\_wait} \cite{manual56} can be used to wait for a certain job or 
number of jobs to complete, which watches the log file and sees if the completion 
entry of the job is made in the log file that is generated by the \emph{log} 
command in the HTCondor submit description file. There are two options we can 
specify to \texttt{condor\_wait}. One is \texttt{-num}, \texttt{number-of-jobs} that waits
till the \texttt{number-of-jobs} are completed and the other option is to specify \texttt{wait}, \texttt{seconds} that 
waits for specified amount of time. If nothing is specified \texttt{condor\_wait} waits indefinitely till
the job(s) are complete.

\section{Implementation of Capacity based scheduling}

This algorithm detects degradation and once we detect degradation we find the 
target capacity:

\begin{algorithm}
\begin{algorithmic}
\STATE $c2 \gets 1.25$ 
\COMMENT {c2: Degradation factor}
\WHILE{true}

\IF{T2 $<$ c2*T1}
  \STATE $jobSubmission(k)$ 
  \COMMENT {Submit k jobs}
  \STATE $T1=(T1+T2)/2$
\ENDIF

\IF{T2 $>$ c2*T1}
  \STATE $degradation\_high \gets k$
  \STATE $degradation\_low \gets k/2$
  \STATE $optimalCapacity(degradation\_high,degradation\_low)$
\ENDIF

\ENDWHILE

\end{algorithmic}
\caption{Algorithm for determining target capacity by detecting degradation}
\label{alg:Degradation Detection}
\end{algorithm}


\begin{algorithm}
\begin{algorithmic}

\STATE $mid \gets (high+low)/2$ 
\STATE $k \gets mid$
\STATE $jobSubmission(k)$ 
  \COMMENT {Submit k jobs}
\IF{T2 $<$ c2 * T1}
\STATE $T1 \gets (T1+T2)/2$
\STATE optimalCapacity(mid,high);
\ENDIF  
\IF{T2 $>$ c2 * T1}
\STATE optimalCapacity(low,mid);
\ENDIF
\RETURN mid
\end{algorithmic}
\caption{Algorithm for determining target capacity by detecting degradation}
\label{alg:optimalCapacity(high,low)}
\end{algorithm}

Algorithm \ref{alg:Degradation Detection} sets up the stage for degradation 
detection. T1 is the turnaround time for previous iteration of job submission
and T2 is the turnaround time for current iteration of job submission. C1 is the 
job propagation factor that is used to vary the amount of jobs submitted and C2 
is the degradation factor that is arbitrarily set to 25\%.
Co-Scheduler submits more jobs if T2 doesn't exceed T1 by the value of C2 * T1. That is, if the turnaround time
hasn't increased by 25\% or else the 
algorithm calls the target capacity algorithm that detects the target capacity.

Algorithm \ref{alg:optimalCapacity(high,low)} is passed two parameters high and 
low. ``high" represents the upper bound where degradation has taken place and ``low" the lower bound 
where degradation hasn't taken place. We have another variable ``mid" which is the average of high and low.
The algorithm submits ``mid" amount of jobs to a site and if the degradation is detected we find capacity in
the lower range of values from low to mid. If the degradation is not detected we find the capacity in 
the upper range of mid to high. The algorithm works like a modified binary 
search and takes O(log n) iterations to find the target set of jobs that a site can handle.

\section{Implementation of Multi-site load distribution based scheduling}
In figure \ref{fig:multiSite1}, multi-site load distribution based scheduling uses the agtime (queue wait time + run-time) at each 
site for sending future jobs. Its a threaded algorithm. The following is the extracted function of per thread execution:

\begin{algorithm}
\begin{algorithmic}

\STATE $c1 \gets 2$ 
\COMMENT {c1: Job propagation factor}
\COMMENT {multiSite is a list storing AgTimes of multiple sites.}
\STATE $low = min(multiSite)$
\STATE $high = max(multiSite)$
\STATE average = $\sum_i multiSite_i$ / Sizeof($multiSite_i$)

\IF{AgTime\_Thread $==$ low}
\STATE $c1 \gets c1 * 4$
\ENDIF

\IF{AgTime\_Thread $==$ high}
\STATE $c1 \gets c1/2>1 ? (c1/2:1)$
\ENDIF

\IF{AgTime\_Thread $<$ average}
\STATE $c1 \gets c1 * 2$
\ENDIF

\end{algorithmic}
\caption{Algorithm for distribution of workflow load across multiple sites on the grid}
\label{alg:updateJobPropagationConstant()}
\end{algorithm}

In Algorithm \ref{alg:updateJobPropagationConstant()}, each thread is responsible for submitting jobs 
to each site. Each thread stores the information of average AgTime of the batch of jobs submitted to the given site.
 Each thread finds the average AgTime from other sites and averages this 
 information to find the average distribution of AgTime across the sites. Once 
 we find this average we compare it with the site average to classify different sites 
into faster and slower sites based on the performance of the sites.

\begin{figure}[htbp!]
\begin{center}
\includegraphics[scale=0.75]{images/multipleSites}
\caption{Multi-site scheduling algorithm overview, classification of sites into slower and faster}
\label{fig:multiSite1}
\end{center}
\end{figure}
\FloatBarrier

\section{Programming APIs}
\subsection{HTCondor Log Reader and User API}

HTCondor provides Job Log Reader API \cite{manual56} that polls logs for job events by giving us 
API access to the events and outcomes. The following is the constructor for 
initializing a ReadUserLog object.

Constructor:
ReadUserLog reader(fp,false,false);
\begin{verbatim}
ULogEventOutcome (defined in condor_event.h):

Status events for job detection:

ULOG_OK: Event is valid
ULOG_NO_EVENT: No event occurred (like EOF)
ULOG_RD_ERROR: Error reading log file
ULOG_MISSED_EVENT: Missed event
ULOG_UNK_ERROR: Unknown Error
\end{verbatim}

All the job log entries are named as events and these events could range from being  \texttt{ULOG\_OK} where 
the event has taken place and is valid to \texttt{ULOG\_UNK\_ERROR} where an error has 
taken place.

The following pseudo-code is extracted from the logReader module that first 
detects all the valid events and then based on the data-structure of the event 
object detects if the event kind is \texttt{ULOG\_EXECUTE}, meaning the job has begun 
executing, via eventNumber data member. Finally, we detect the \texttt{ULOG\_JOB\_TERMINATED} 
event where the job has successfully terminated. We also cast the more general event 
object into JobTerminatedEvent to access data members of the 
JobTerminatedEvent. The comments in the listed pseudo-code provides more details 
on this extracted code.


Job Submission Pseudo-Code:

\begin{lstlisting}
void logReader(string hostFile, args *data, int nSites)	{		
		FILE *fp;
		ReadUserLog reader(fp,false,false);
		ULogEvent *event = NULL;		
             while(reader.readEvent(event)==ULOG_OK)	{                    
            if((*event).eventNumber==ULOG_EXECUTE )	{                                
            //Cast into Execute Event
                ExecuteEvent *exec 
                = static_cast<ExecuteEvent*>(event);                                       
                //condor_wait -num K, where K is the 
                //amount of jobs completed till the wait.
                    char tmp[100];
                    sprintf(tmp,"condor_wait -num 
                    \%d \%s",count,hostFile.c_str());
            }            
            if((*event).eventNumber==ULOG_JOB_TERMINATED)	{                
            //Cast into Job Terminated Event
                JobTerminatedEvent *term 
                = static_cast<JobTerminatedEvent*>(event);                
                
                if(term->normal)	{                    
                    //on Normal termination, 
                    //works only for local jobs, 
                    //find the CPU time of local jobs
                }               
            }
        }
}

\end{lstlisting}
\subsection{Synchronization of  Co-Scheduler Code}

There are critical sections in the code where synchronization becomes absolutely 
necessary. One shared variable is number-of-jobs. The number of jobs executed across 
the sites should remain fixed and the value must match the value that has been 
given as input. In a threaded system where each thread is executing jobs on a 
different cluster it becomes necessary to define \texttt{N}, the number of jobs executed or executing as a 
critical section. Here we define a pthread mutex and lock it for all write accesses to 
\texttt{N}. We serialize the access to \texttt{N} and make conditional checks during job submission, so as not to allow
job submissions when \texttt{N} is greater than the input value. By 
serializing the access across multiple threads the total jobs executed remains 
equal to the input number of jobs. The following code block 
demonstrates the use of mutexes for serialization of  \texttt{N} among the threads.

\begin{figure}[htbp!]

\begin{lstlisting}
pthread_mutex_t mymutex = PTHREAD_MUTEX_INITIALIZER;
pthread_mutex_lock (&mymutex);
sumOfK+=k;
pthread_mutex_unlock (&mymutex);
\end{lstlisting}

\caption{Mutex on Number Of Jobs, sumOfK variable}
\label{fig:mutex}

\end{figure}


Another section of code where synchronization becomes important is while 
distributing the jobs to multiple sites. When measuring which cluster is 
faster, we need information of turnaround time from all the sites 
before proceeding. This means all the threads need to be executing the same line of code 
before proceeding with the further program. Thus, the absolute need for 
synchronization. To handle this problem we make use of a conditional pthread variable. The following code demonstrates the use of conditional wait 
and conditional signal variable that clears the block on all the threads waiting 
based on the given condition.


\begin{figure}[htbp!]

\begin{lstlisting}
pthread_mutex_t syncMutex = PTHREAD_MUTEX_INITIALIZER;
pthread_cond_t synchronize_cv = PTHREAD_COND_INITIALIZER;

pthread_mutex_lock(&syncMutex);    
multiSite.push_back(stats[thread_id].T1);	
tid_multiSite.insert(std::pair<int,int>
(data->tid,stats[thread_id].T1)
);
	   
if ( multiSite.size() < numOfSites )
{
  pthread_cond_wait(&synchronize_cv, &syncMutex);
}
else
{		
pthread_cond_broadcast(&synchronize_cv);
}
pthread_mutex_unlock(&syncMutex);
    
\end{lstlisting}
\caption{Synchronization of threads after reading turnaround time}
\label{fig:synchronization}
\end{figure}
\FloatBarrier

We need to have the turnaround time of the first job,  which enables us to measure 
how fast each cluster is. To do so, we need to wait for each job to complete on 
the first submission thereafter the jobs are submitted asynchronously and the 
average turnaround time is updated in the list. In this pseudocode the threads 
beginning first add the turnaround time to the list and conditionally wait if 
the size of the list is less than the number of sites. The last thread comes in 
and the same condition is voided and executes a conditional broadcast to unblock 
all the waiting threads and the scheduling proceeds further to asynchronously schedule further 
jobs.
In this example of synchronization code there is a possibility that all of the 
threads wait continuously if the availability of the sites is very low. The 
above code needs availability on clusters to prevent threads waiting for long time. The 
Co-Scheduler will wait till the timeout period and if the jobs do not execute we 
remove the jobs on that site and proceed further with Co-Scheduling thereby preventing a deadlock.
\chapter{Evaluation}

\section{Introduction}
We evaluate the degradation detection algorithm by the distribution of turnaround time. The turnaround 
time of all the jobs when run using co-scheduler and the turnaround time of all 
the jobs when run directly on the cluster using the HTCondor-G grid interface were both investigated.

If a NFS server has the capacity to serve 50 clients without degradation it
implies the turnaround time of all the jobs on the 50 clients are within the 
limits that would not be a degraded turnaround time. If the same NFS server is forced to serve 200 
clients then we would see a degradation and would see an increase in the turnaround time of the jobs
that are now contending for the I/O. This would result in degraded turnaround time and 
the same jobs would have to take lot more time to complete which is the usual 
scenario that occurs at Holland Computing Center when large amount of I/O bound 
jobs are submitted to the cluster and there is no check on the usage and the 
capacity of the I/O resource.

In this evaluation, we've submitted 375 jobs to the cluster ``Tusker" using bulk 
HTCondor-G submission and then submitting it through the co-scheduler so that we 
can keep track of degradation, detect it and prevent it. The submit scripts for 
the co-scheduler are generated dynamically and the grid universes are populated 
from the input sites file. The machine used for jobs submission was a 
HTCondor developmental virtual machine, that had to be configured for the grid 
submissions before its use for the experiment. 375 Jobs were run on the lustre 
filesystem and the results for this run are in figure \ref{fig:tusker_histogram} is 
of ``Tusker". We were able to see degradation when we 
submitted jobs in bulk using the HTCondor-G interface but we couldn't be sure if the 
degradation was the result of just our I/O based test job. Thus, separate 
individual NFS servers were setup on both the clusters sandhills and tusker and 
tests were run again. The results of the tests are presented in figure \ref{fig:nfs_tusker_histogram} 
and figure \ref{fig:nfs_coscheduler_tusker_histogram}.

We're making use of histograms of binned turnaround time and good 
put graphs to study the effect of degradation. A histogram that projects 
degradation has many jobs that have higher turnaround times compared to a non-degraded graph. A histogram that doesn't reflect degradation will have most of 
its jobs within the permissible limits of turnaround time and won't have large 
variance in terms of the turnaround time. We also make use of goodput graphs 
that plot an area graph w.r.t the sorted list of turnaround time. This provides a good way to 
visualize the turnaround time. Ideally, the area under the curve must be minimum 
in case of a non degraded graph but in a graph that projects degradation the 
area under the curve is larger than its non-degraded counterpart.
\newpage
\section{Evaluation of Co-Scheduler on Tusker cluster with Lustre}
Figure \ref{fig:tusker_histogram} shows how there exists a large variance in 
turnaround time by having a larger turnaround time than that present in figure 
\ref{fig:coscheduler_histogram}. The former was the result of bulk submission 
whereas the later is the result of a co-scheduler run. The associated goodput 
graphs in figure \ref{fig:tusker_jobgoodput} and figure \ref{fig:coscheduler_jobgoodput} 
show their respective areas with degraded run taking up larger area. When we look 
at the turnaround time distribution on independent NFS servers we can infer many 
things, that the degradation is significant in the figure \ref{fig:nfs_tusker_histogram} and the co-scheduler 
resolves degradation to a larger extent as noted in figure 
\ref{fig:nfs_coscheduler_tusker_histogram}.

\begin{figure}[htbp!]

\begin{lstlisting}

int m,n,i,j;

n = atoi(argv[1]);
n = n*100;
m = n+1;
ofstream ofile(argv[2],ios::app);

for(j=0;j<m;j++)	{
	for(i=0;i<m/100;i++)	{
ofile<<i<<j;
}
ofile.flush();
}

ofile.close();


\end{lstlisting}
\caption{The program that generates I/O load for a given period of time}
\label{fig:ioload}
\end{figure}
\FloatBarrier


\begin{figure}[h!]
\begin{center}
\includegraphics{images/tusker_histogram}
\caption{Histogram showing turnaround time distribution of 375 Jobs, when run on Tusker cluster via bulk $Condor\_G$ submission}
\label{fig:tusker_histogram}
\end{center}

\begin{center}
\includegraphics{images/coscheduler_histogram}
\caption{Histogram showing turnaround time distribution of 375 Jobs, when run on Tusker  via Co-Scheduler}
\label{fig:coscheduler_histogram}
\end{center}


\end{figure}
\FloatBarrier

\section{Evaluation of Co-Scheduler on independent NFS server}

The test I/O job used in this experiment continuously does disk write operations for a given period of time, the 
usual time taken by the program being 300-500 seconds when run on different resources. It generated approximately 3 GB of data file
in that period. Designing our own test I/O program,  facilitated the change of the path
to the NFS servers where we'd be doing our benchmark to test the servers. The test I/O job takes two 
parameters, 
one is the size N and the other is the path where the I/O must take place. There are two 
clusters under consideration, Sandhills and Tusker. 375 jobs are run on each of 
the clusters and then a separate set of 375 jobs are scheduled using 
Co-Scheduler. We measure the extent of degradation on each cluster and quantitatively 
determine how degradation is handled by the Co-Scheduler.

In figure \ref{fig:tusker_histogram}  considerable jobs take more than 2500 seconds to complete. 
When the resource is not degraded the test job takes approximately 300-500 seconds of time. Tusker is 
currently running Lustre filesystem. If the capacity of the filesystem permitted 
to serve 375 jobs (or the number of jobs running concurrently) then most jobs 
would be completed within 500 seconds of time but that isn't the case with Tusker. We can conclude the resultant graph is the cause of degradation 
of the disk I/O resource, but there needs to be clarification about the source of the degradation. 
It could either be caused by the test I/O jobs that were submitted or can be caused 
by a set of jobs that are already running on the cluster. To clarify this issue 
and get a clear picture of what caused the degradation the result from the 
figure \ref{fig:nfs_tusker_histogram} sheds more information.


\begin{figure}[h!]
\begin{center}
\includegraphics{images/tusker_jobgoodput}
\caption{Goodput graph showing the turnaround time distribution of all the jobs submitted on Tusker via $Condor\_G$ bulk submission}
\label{fig:tusker_jobgoodput}
\end{center}

\begin{center}
\includegraphics{images/coscheduler_jobgoodput}
\caption{Goodput graph showing the turnaround time distribution of 375 jobs submitted on Tusker with the Co-Scheduler}
\label{fig:coscheduler_jobgoodput}
\end{center}

\end{figure}
\FloatBarrier

\begin{figure}[htbp!]
\begin{center}
\includegraphics{images/nfs_tusker_histogram}
\caption{Histogram showing turnaround time distribution of 300 Jobs, when run via $Condor\_G$ bulk submission on a custom small scale independent NFS server on Tusker}
\label{fig:nfs_tusker_histogram}
\end{center}

\begin{center}
\includegraphics{images/nfs_coscheduler_tusker_histogram}
\caption{Histogram showing turnaround time distribution of 300 Jobs, when run using a Co-Scheduler on a custom small scale independent NFS server on Tusker}
\label{fig:nfs_coscheduler_tusker_histogram}
\end{center}


\end{figure}
\FloatBarrier

\begin{figure}[htbp!]

\begin{center}
\includegraphics{images/nfs_tusker_goodput}
\caption{Goodput graph showing the turnaround time distribution of 300 Jobs, when run via $Condor\_G$ bulk submission on a custom small scale independent NFS server on Tusker}
\label{fig:nfs_tusker_goodput}
\end{center}


\begin{center}
\includegraphics{images/nfs_coscheduler_tusker_goodput}
\caption{Goodput graph showing the turnaround time distribution of 300 jobs submitted, when run using a Co-Scheduler on a custom small scale independent NFS server on Tusker}
\label{fig:nfs_coscheduler_tusker_goodput}
\end{center}


\end{figure}
\FloatBarrier



In the figures \ref{fig:nfs_tusker_goodput} and \ref{fig:nfs_coscheduler_tusker_goodput} we can see most of the jobs take about 
500 seconds of time to complete. We've prevented excessive degradation here and 
the I/O resources serve the concurrent jobs without a degraded turnaround time. 
This reduces the total time taken for the execution of this batch of jobs 
making way for the use of cluster for other jobs that might need computing.

A Multi-Site load distribution algorithm distributes the job based on agtime (queue wait time + run-time) and load on the cluster. The following was the distribution of jobs when run 
on Co-Scheduler with 750 Jobs. The cluster with lower AgTime will get more amount of jobs to run. 
For Tusker the throughput was 47.2 CPU time hours/elapsed time hours and for Sandhills it was 3.12 CPU time hours/elapsed time hours. 
Which means there was greater availability on Tusker and more jobs were 
submitted to tusker than Sandhills.

\section{Scalability Analysis of the Co-Scheduler}

The Co-Scheduler is architected in a way so that each thread is responsible for 
dispatching jobs on a given site. The Co-Scheduler can potentially spawn thousands 
of threads and serve them efficiently but there are only hundreds of OSG sites 
to run on. It can certainly scale to all the sites of the OSG. 

Co-Scheduler can take a large number of jobs as input, of the order of hundreds 
of thousands of jobs and dispatch the jobs efficiently with the load-balancing 
algorithm and ensure these jobs run without incurring degradation.

Co-Scheduler is a robust program written in C++ that schedules based on 
turnaround time to detect degradation and schedules based on  (queue wait time + 
run-time) for the sake of load balancing.

\chapter{Conclusion}

Co-Scheduling on the Grid is challenging with the presence of dynamically varying 
resources like Network, RAM, compute resources, all of which are available opportunistically. 
The current Co-Scheduler 
detects degradation and taxes degraded resources less in terms of scheduling 
jobs to the degraded resources. The capacity of the cluster is a varying 
quantity that determines the amount of concurrent jobs that may run without 
degradation. Co-Scheduler successfully finds the capacity of the cluster and 
maintains the capacity upto 10 successive iterations of the scheduler before 
recalculating the capacity.

Scheduling based on agtime enables us to submit more jobs to the clusters having greater availability. Co-Scheduler improves throughput of the job set which handles the 
case of degradation by virtue of a multi-site job distribution algorithm. Because Co-Scheduler directly effects the load on the resource, Co-Scheduler 
ensures the average overall turnaround time per job is lower.

Resource management is completely abstracted. We do not need to consider the 
allocation and management of disk I/O or RAM or Network I/O. The Co-Scheduler is 
driven by the fact that whenever a resource is overloaded it results in 
degradation and subsequently does not respond optimally or crashes. A simple and 
efficient way of tracking degradation is implemented which monitors the load of 
resources on the cluster. As much as this is a positive aspect of the 
Co-Scheduler, the downside is that there isn't a way to tell which of the resource is 
the cause of the degradation. Because the Co-Scheduler is used with large workloads is usually known based on 
the workload the resource that is under stress.

While Co-Scheduling offers multiple enhancements for the users of large workloads 
there are some downsides of it. Co-Scheduler submits the jobs stepwise, in 
the sense that it submits a smaller set of jobs and waits for its outcome to decide upon the quantity 
of jobs on the next iteration. This 
nature of the Co-Scheduler can accumulate a lot of queue wait time if the availability 
of the compute resources are lower. In the multi-site load distribution 
algorithm, if one of the sites has very low availability the Co-Scheduler would 
accumulate the queue wait time of that particular site before it can complete 
its execution. To solve this issue we schedule based on AgTime (queue wait time + run-time) so that clusters with availability 
get more number of jobs than the ones with lower availability. When a batch of 
jobs is submitted to multiple grid sites, the sites not responding timeout after 
a defined period of time and clean up the jobs that were submitted so that the Co-Scheduler may further 
schedule these jobs on faster and more available clusters. 

\section{Future work}

Degradation factor, C2 plays an important role in the detection of degradation and in determining the capacity of the cluster. An important future
work is to derive C2 specifically for a given resource. It would be of importance for scheduling if we had the ability to classify the nature of degradation either by taking input from the user or detect dynamically from a degraded resource at run-time.

To protect the resources at HCC its necessary to enforce Co-Scheduler like scheduling mechanism. To enforce Co-Scheduler like scheduling on protected resources, further work needs to be done to integrate Co-Scheduler code in the local resource manager. Another important aspect would be to include the features of Co-Scheduler in Job Router. 
Further study of Co-Scheduler's effect on throughput would provide us directions to tweak multi-site load balancing algorithm.

%% backmatter is needed at the end of the main body of your thesis to
%% set up page numbering correctly for the remainder of the thesis
\backmatter

%% Start the correct formatting for the appendices
\appendix

%% Appendices go here (if you have them)

%% Bibliography goes here (You better have one)
%% BibTeX is your friend
%% Index go here (if you have one)
%% Bibliography goes here (You better have one)
%% BibTeX is your friend
\bibliographystyle{plain}
\bibliography{KartikThesis}
%% Pull in all the entries in the bibtex file. Is is a useful trick to
%% check all your references.
\nocite{*}

%% Index go here (if you have one)

\end{document}